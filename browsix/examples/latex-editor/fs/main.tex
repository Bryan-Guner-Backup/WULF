\documentclass[11pt]{article}
\usepackage{cite}
\usepackage[T1,hyphens]{url}

\newcommand{\Browsix}{\textsc{Browsix}}

\begin{document}

\title{\bf \Browsix{}: Bringing Unix to the Browser}
\author{The \Browsix{} Authors}
\date{Today}
\maketitle

\begin{abstract}

While standard operating systems like Unix make it relatively simple
to build complex applications, web browsers lack the features that
make this possible. In this paper, we present \Browsix{}, a
JavaScript-only framework that brings the essence of Unix to the
browser. \Browsix{} makes core Unix features available to web
applications (including pipes, processes, signals, sockets, and a
shared file system) and extends JavaScript runtimes for C, C++, Go,
and Node.js programs so they can run in a Unix-like environment within
the browser. We illustrate \Browsix{}'s capabilities by converting a
client-server application to run entirely in the browser and
developing a serverless \LaTeX{} editor that executes PDFLaTeX and
BibTeX in the browser.

\end{abstract}

\section{Related Work}
\label{sec:related}

\Browsix{} builds on and extends its filesystem component,
BrowserFS. Emscripten compiles LLVM bytecode to JavaScript, enabling
the compilation of C and C++ to JavaScript~\cite{emscripten}.
\Browsix{} augments its runtime system so that unmodified C and C++
programs compiled with Emscripten can take full advantage of its
facilities. GopherJS compiles Go code to JavaScript and provides
similar runtime support~\cite{musiol:2016gopherjs}.

\bibliography{mybib}{}
\bibliographystyle{plain}
\end{document}
